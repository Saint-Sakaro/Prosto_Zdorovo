\documentclass[a4paper,12pt]{article}

\usepackage[utf8]{inputenc}
\usepackage[russian]{babel}
\usepackage[T2A]{fontenc}
\usepackage{geometry}
\geometry{margin=2cm}
\usepackage{setspace}
\onehalfspacing

\title{Техническое задание: Модуль геймификации веб-приложения «Карта здоровья»}
\author{}
\date{}

\begin{document}

\maketitle

Настоящее техническое задание описывает модуль геймификации и мотивации пользователей для веб-приложения «Карта здоровья», разрабатываемого на стеке Django (backend) и React (frontend). Цель модуля геймификации состоит в том, чтобы стимулировать пользователей активно и регулярно оставлять качественные, максимально информативные и по возможности уникальные отзывы об объектах инфраструктуры, а также фиксировать инциденты в городской среде, одновременно снижая уровень спама и дублирующегося контента.

Веб-приложение «Карта здоровья» предназначено для отображения на интерактивной карте объектов инфраструктуры, связанных со здоровым образом жизни и состоянием городской среды. Пользователь может отмечать на карте точки, такие как магазины с полезными продуктами, места продажи алкоголя и табачных изделий, аптеки, поликлиники, клиники, спортивные объекты и другие организации, оставлять отзывы, прикреплять фотографии, просматривать рейтинги объектов и участвовать в системе геймификации. Приложение должно иметь современный, нестандартный и визуально привлекательный интерфейс, ориентированный на молодую аудиторию и жюри хакатона, с элементами игровой стилистики, динамичными акцентами, прогресс-барами, значками, уровнями и аккуратной интеграцией с картой. [web:1][web:3]

Модуль геймификации разделяет три ключевых показателя активности пользователя: общий рейтинг, месячный рейтинг и баллы. Общий рейтинг (Reputation, глобальный рейтинг) представляет собой накопительный показатель авторитета пользователя в системе. Он не обнуляется со временем и растёт за счёт подтверждённых полезных действий: создания уникальных отзывов, фиксации важных инцидентов, вклада в качество данных и другие действия, предусмотренные системой. Общий рейтинг влияет на вес голоса пользователя при оценке объектов и на его положение в глобальной таблице лидеров.

Месячный рейтинг (Monthly Leaderboard) отражает активность пользователя за текущий календарный месяц. Этот показатель обнуляется автоматически первого числа каждого месяца. Месячный рейтинг используется для формирования сезонной таблицы лидеров и определения пользователей, которые получают ежемесячные поощрения и призы от государства, муниципалитета и партнёров проекта. Высокое место в месячном рейтинге мотивирует пользователей возвращаться в приложение и поддерживать постоянную активность.

Баллы (Points) являются внутренней виртуальной валютой системы. Баллы начисляются за подтверждённые полезные действия, такие как добавление уникальных отзывов, фиксация значимых инцидентов, прикрепление доказательных материалов (фотографий и видео), а также иные действия, определённые правилами геймификации. Баллы могут тратиться пользователем в маркетплейсе наград на скидочные купоны, цифровой и реальный мерч, а также специальные привилегии. В конце месяца баллы либо сгорают, либо частично конвертируются в опыт или дополнительный вклад в общий рейтинг, в соответствии с настроенными коэффициентами, что стимулирует пользователя использовать их в течение месяца, а не накапливать бесконечно.

Отзывы в системе делятся на два основных типа сущностей. Первый тип — это отзывы о статических объектах (POI, Point of Interest), привязанных к конкретному зданию или бизнесу: рестораны, магазины, спортивные залы, кинотеатры, аптеки, медицинские центры и другие подобные объекты. Основная цель таких отзывов — оценка качества услуг, доступности, соответствия принципам здорового образа жизни, а также общее впечатление пользователя от посещения объекта. Второй тип — это динамические точки (Incident Points), которые отражают фиксируемые пользователями события или проблемы в городской среде: разбросанный мусор, разрушенная инфраструктура, ямы, неполадки, факты нелегальной торговли, небезопасные зоны и другие инциденты. Такие точки создаются «здесь и сейчас» и служат для оперативной фиксации проблем, требующих внимания.

Для каждой создаваемой записи (отзыва или инцидента) система должна реализовать алгоритм автоматической проверки уникальности перед начислением значимых наград. При отправке нового отзыва система анализирует его параметры: географические координаты, категорию и временное окно. Реализуется поиск существующих записей в радиусе \(R\) метров (например, 50 метров) и во временном интервале \(T\) (например, 24 часа) с той же или близкой категорией. Если в указанном радиусе и интервале уже присутствуют схожие по категории активные метки, новый отзыв классифицируется как дубликат или подтверждение популярного инцидента. В этом случае отзыв автоматически принимается без отправки на дополнительную модерацию, пользователю начисляется минимальное фиксированное количество баллов, а влияние на общий рейтинг либо отсутствует, либо является незначительным. Таким образом пользователь поощряется за подтверждение уже известной информации, но не получает чрезмерной выгоды за массовое создание дубликатов.

Если в радиусе \(R\) и в окне \(T\) не обнаруживается активных меток той же категории, отзыв рассматривается как кандидат на уникальность. В этом случае запись помечается специальным флагом и переводится в статус ожидания модерации. Для таких отзывов значимые награды откладываются до момента принятия решения модератором. Модерация может осуществляться человеком-модератором или доверенным AI-агентом, отображающим в интерфейсе все необходимые данные: текст отзыва, тип (POI или инцидент), координаты, время создания, прикреплённые медиафайлы, а также историю активности пользователя.

Для процесса модерации предусмотрено три основных сценария. При выборе модератором действия «Подтвердить» отзыв получает статус approved, фиксируется как уникальный значимый вклад в карту здоровья, а пользователю начисляются максимальные баллы виртуальной валюты и значимое увеличение общего рейтинга. Наличие фото- или видеоматериалов выступает как валидирующий фактор и может увеличивать коэффициент награждения, что дополнительно мотивирует пользователей прикреплять доказательства. При выборе действия «Не уверен» или «Неактуально» отзыв получает статус soft\_reject, баллы и рейтинг не изменяются, а пользователю не назначаются ни награды, ни штрафы. Это нужно для ситуаций, когда данных недостаточно или проблема уже исчезла, но намерений злоупотребления нет. При выборе статуса «Спам» или «Фейк» отзыв помечается как spam\_blocked, баллы не начисляются, а общий рейтинг пользователя уменьшается, что создаёт ощутимый штраф за попытку обмана системы. При накоплении нескольких спамовых инцидентов возможна автоматическая временная блокировка учётной записи.

Формально модуль геймификации должен хранить необходимую информацию в базе данных и предоставлять стандартизированные интерфейсы доступа для остальных частей системы. Для пользователя должны храниться идентификатор (например, UUID), имя или логин, общий рейтинг, месячный рейтинг, текущий баланс баллов, уровень, а также список полученных достижений. Общий рейтинг не сбрасывается и отражает долгосрочный вклад, месячный рейтинг обнуляется раз в месяц, а баланс баллов используется для покупки наград. Уровень и достижения могут вычисляться на основе комбинации рейтинга, количества подтверждённых уникальных отзывов и иных метрик активности.

Для сущности отзыва необходимо хранить идентификатор, тип (poi\_review или incident), координаты точки, категорию, текстовое содержание, флаг наличия медиа-доказательств, ссылку на автора, признак уникальности, статус модерации и временные метки создания и изменения. Статус модерации должен принимать значения pending (ожидает проверки), approved (подтверждён), soft\_reject (неактуален или недостаточно подтверждён), spam\_blocked (признан спамом или фейком). Признак уникальности устанавливается после выполнения автоматической проверки на дубли и может уточняться по результатам модерации.

Для операций начисления и расходования баллов рекомендуется ввести сущность транзакции наград с полями: пользователь, количество баллов, причина начисления или списания, тип операции и временная метка. Это позволит прозрачно отслеживать историю изменений, строить отчёты и отлаживать логику геймификации. Причинами начисления могут быть, например, подтверждение уникального отзыва, участие в сезонной активности, корректное дополнение информации об объекте. Причинами списания — покупка наград в магазине, конвертация части баллов в опыт или рейтинг в момент ежемесячного сброса и другие операции, определённые бизнес-логикой.

Система геймификации должна включать маркетплейс наград, в котором пользователь может тратить накопленные баллы. Каталог наград может содержать скидочные купоны от партнёров (магазины спортивного питания, фитнес-клубы, сервисы здорового питания), цифровой мерч в виде значков профиля, рамок аватаров, особых статусов и званий, а также реальные призы и сувениры. Цены наград выражаются в баллах, а для каждой транзакции обмена необходимо проверять достаточность баланса пользователя и уменьшать его при успешном обмене. Модуль должен поддерживать хранение каталога наград, текущего статуса доступности каждой награды, а также статистику востребованности призов.

Месячный рейтинг используется в системе сезонных наград. В конце каждого календарного месяца формируется таблица лидеров по месячному рейтингу, обычно по каждому городу или региону отдельно. Топ-\(N\) пользователей (например, топ-10) могут получать специальные призы от государственных и муниципальных органов: грамоты, приглашения на мероприятия, билеты на события, встречу с администрацией и другие формы нематериального и материального поощрения. Формирование этой таблицы должно поддерживать фильтрацию по территории и корректную обработку равных результатов, а также возможность публикации итогов и отправки уведомлений участникам.

Жизненный цикл геймификации включает ежемесячный сброс отдельных показателей. Первого числа каждого месяца в заданное время (например, в полночь по московскому времени) для всех пользователей производится обнуление значения месячного рейтинга. Баланс баллов может либо обнуляться полностью, либо частично конвертироваться в дополнительный вклад в общий рейтинг или опыт, согласно заранее заданному коэффициенту. При этом общий рейтинг и полученные достижения остаются неизменными, чтобы сохранить ощущение долгосрочного прогресса. После выполнения сброса пользователям можно отправлять уведомления о начале нового сезона и возможности вновь побороться за лидерство.

Для полного функционирования модуля геймификации необходимо реализовать набор прикладных интерфейсов. Должны быть доступны эндпоинты для получения глобальной таблицы лидеров по общему рейтингу, получения сезонной таблицы лидеров по месячному рейтингу, просмотра профиля пользователя с его показателями и достижениями, создания нового отзыва с автоматической проверкой уникальности, выполнения действий модератора над отзывом, просмотра каталога наград и обмена баллов на выбранные призы. Эти эндпоинты должны быть спроектированы в соответствии с REST-подходом и интегрированы с фронтендом на React.

Пользовательский интерфейс модуля геймификации должен включать экран профиля пользователя с отображением аватара, текущего места в глобальном и месячном рейтингах, значений общего рейтинга, месячного рейтинга и баланса баллов, прогресс-бара до следующего уровня, списка полученных достижений и истории последних отзывов со статусами модерации. Таблицы лидеров должны показывать список пользователей с их позициями, основными метриками и возможностью переключения между глобальным и месячным режимами. Маркетплейс наград должен представлять собой каталог карточек с описанием награды, ценой в баллах и кнопкой обмена.

Интерфейс модератора должен предоставлять список отзывов в статусе ожидания проверки, возможность фильтрации по типам отзывов, времени, наличию медиа-доказательств и активности автора, а также три основные управляющие действия: подтверждение, мягкий отказ и пометка как спам. После выполнения действий модерации необходимо обновлять показатели пользователя и статусы отзывов, а также при необходимости уведомлять пользователя о результате проверки и начисленных или не начисленных наградах.

Модуль геймификации должен быть тесно интегрирован с остальными подсистемами приложения «Карта здоровья», включая модуль оценки «здоровости» географических точек, систему авторизации и профилей, модуль хранения и отображения отзывов, карту и панель администратора. Данные о рейтингах, баллах и статусах отзывов могут использоваться как дополнительные факторы при расчёте индекса «здоровости» районов и при формировании доверия к тем или иным источникам пользовательского контента. Вся логика геймификации должна быть реализована с учётом возможности масштабирования, расширения набора правил и подключения внешних партнёрских программ поощрения.

\end{document}
