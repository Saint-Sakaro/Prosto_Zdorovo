\section{Модуль рейтинга объектов (0--100), динамических анкет и влияния отзывов}

\subsection{Цель модуля}

Целью модуля является расчёт \textbf{Индекса пользы (Health Impact Score, HIS)} для объектов городской инфраструктуры в диапазоне от 0 до 100. Система должна:

\begin{itemize}
  \item опираться на статические характеристики объекта (анкета по типу места);
  \item корректировать рейтинг с учётом динамических отзывов пользователей;
  \item поддерживать разные типы объектов с разными наборами признаков, но приводить их к общей шкале 0--100;
  \item обеспечивать пересчёт рейтинга при добавлении, изменении и удалении отзывов.
\end{itemize}

\subsection{Типы объектов и динамическая анкета}

Поддерживаемые типы объектов (может расширяться):

\begin{itemize}
  \item точки сбора мусора;
  \item промышленные предприятия;
  \item рекламные конструкции (наружная реклама);
  \item предприятия общественного питания;
  \item магазины;
  \item спортивные объекты;
  \item остановки общественного транспорта;
  \item медицинские организации.
\end{itemize}

\subsubsection{Dynamic Schema (анкета по типу места)}

Для каждого типа объекта хранится \textbf{JSON-схема анкеты} --- список полей с типами, весами и знаками влияния.

Поддерживаемые типы полей:

\begin{itemize}
  \item \texttt{boolean} --- есть/нет (освещение, урна, доступность);
  \item \texttt{range} --- числовой диапазон (состояние 1--5, расстояние в метрах и т.п.);
  \item \texttt{select} --- выбор из списка значений (тип покрытия, класс опасности и т.п.);
  \item \texttt{photo} --- наличие медиа-подтверждения (влияет на доверие/верификацию, но не даёт прямого балла).
\end{itemize}

Для каждого поля в схеме задаются, как минимум, следующие атрибуты:

\begin{itemize}
  \item \texttt{id} --- строковый идентификатор признака;
  \item \texttt{type} --- тип (\texttt{boolean|range|select});
  \item \texttt{direction} --- направление влияния: \(+1\) (полезный признак) или \(-1\) (вредный признак);
  \item \texttt{weight} --- вес признака \(w_i\) (важность в инфраструктурном рейтинге);
  \item \texttt{scale\_min}, \texttt{scale\_max} --- границы диапазона для нормализации в \([0;1]\) (для \texttt{range});
  \item \texttt{mapping} --- таблица соответствий значений \(\rightarrow [0;1]\) (для \texttt{select}).
\end{itemize}

\subsubsection{Генерация анкеты новой категории через LLM}

При добавлении новой категории (новый тип места) вызывается LLM, которая по названию категории и базовому описанию:

\begin{itemize}
  \item генерирует список полей анкеты (тип, текст вопроса, направление влияния);
  \item предлагает начальные веса \(w_i\) и возможные значения \texttt{select}-полей;
  \item возвращает JSON-схему, которую администратор просматривает, корректирует и утверждает.
\end{itemize}

\subsection{Статический рейтинг объекта (Infrastructure Score)}

Обозначения:

\begin{itemize}
  \item \(x_i\) --- фактическое значение \(i\)-го признака по анкете;
  \item \(s_i \in [0;1]\) --- нормализованный вклад признака;
  \item \(w_i\) --- вес признака (может быть положительным или отрицательным);
  \item \(S_{\text{infra}} \in [0;100]\) --- статический инфраструктурный рейтинг объекта.
\end{itemize}

\subsubsection{Нормализация признаков}

Для каждого признака вычисляется \(s_i \in [0;1]\).

\paragraph{Boolean-признак.}

\begin{itemize}
  \item Если \texttt{direction = +1} (признак полезный):
  \begin{itemize}
    \item \texttt{true} \(\rightarrow s_i = 1\);
    \item \texttt{false} \(\rightarrow s_i = 0\).
  \end{itemize}
  \item Если \texttt{direction = -1} (признак вредный):
  \begin{itemize}
    \item \texttt{true} \(\rightarrow s_i = 0\);
    \item \texttt{false} \(\rightarrow s_i = 1\).
  \end{itemize}
\end{itemize}

\paragraph{Range-признак.}

Для полезного параметра (чем больше, тем лучше):

\[
s_i = \frac{x_i - \text{min}_i}{\text{max}_i - \text{min}_i}.
\]

Для вредного параметра (чем больше, тем хуже):

\[
s_i = 1 - \frac{x_i - \text{min}_i}{\text{max}_i - \text{min}_i}.
\]

\paragraph{Select-признак.}

Для каждого возможного значения в справочнике задаётся заранее \(s_i \in [0;1]\).  
Например:
\begin{itemize}
  \item «раздельный сбор отходов есть» \(\rightarrow s_i = 1\);
  \item «только общий контейнер» \(\rightarrow s_i = 0.3\);
  \item «контейнеров нет» \(\rightarrow s_i = 0\).
\end{itemize}

\subsubsection{Веса и инвертированная логика для рисковых объектов}

\begin{itemize}
  \item Для полезной инфраструктуры (спорт, медицина, ЗОЖ) веса \(w_i > 0\) для «хороших» признаков (безопасное покрытие, освещение, доступность МГН и т.д.).
  \item Для объектов риска (алкомаркеты, табак, опасные производства) вводится инвертированная логика:
  \begin{itemize}
    \item признаки «соблюдение ограничений» (удалённость от школ, отсутствие рекламы, ограниченный режим работы) дают положительный вклад;
    \item признаки «нарушения» (реклама алкоголя, близость к школам, круглосуточная продажа) дают отрицательный вклад;
    \item итоговый балл 0--100 трактуется как: низкое значение --- зона повышенного риска, высокое --- формально безопасный объект.
  \end{itemize}
\end{itemize}

\subsubsection{Формула инфраструктурного рейтинга}

Пусть имеется \(N\) признаков с весами \(w_i\) и нормализованными значениями \(s_i\).  
Вычисляется:

\[
S_{\text{infra\_raw}} = \frac{\sum_{i=1}^{N} w_i \cdot s_i}{\sum_{i=1}^{N} |w_i|}.
\]

Приведение к шкале 0--100:

\[
S_{\text{infra}} = \min(100, \max(0, 100 \cdot S_{\text{infra\_raw}})).
\]

\subsection{Динамический рейтинг по отзывам (Social Score)}

Пусть:

\begin{itemize}
  \item \(r_j \in [1;5]\) --- оценка \(j\)-го отзыва (звёзды);
  \item \(t_j\) --- время создания отзыва;
  \item \(R_{\text{author}(j)}\) --- репутационный статус автора;
  \item \(w_{\text{time},j}\) --- временной коэффициент (time decay);
  \item \(w_{\text{author},j}\) --- вес по репутации автора;
  \item \(S_{\text{social}} \in [0;100]\) --- динамический рейтинг по отзывам.
\end{itemize}

\subsubsection{Time Decay (снижение веса старых отзывов)}

Задаётся период полураспада \(T_{1/2}\) (например, 180 дней).  
Пусть отзыв имеет возраст \(\Delta t_j\) дней. Тогда:

\[
w_{\text{time},j} = 2^{-\Delta t_j / T_{1/2}}.
\]

\subsubsection{Вес по репутации автора}

Примерные значения (конфигурируемые):

\begin{itemize}
  \item Новичок: \(w_{\text{author},j} = 0.5\);
  \item Активный житель: \(w_{\text{author},j} = 1.0\);
  \item Эксперт: \(w_{\text{author},j} = 1.5\).
\end{itemize}

\subsubsection{Нормализация оценки отзыва и итоговая формула}

Нормализация оценки отзыва в диапазон \([0;1]\):

\[
s^{(\text{rev})}_j = \frac{r_j - 1}{5 - 1}.
\]

Вес отзыва:

\[
w_j = w_{\text{time},j} \cdot w_{\text{author},j}.
\]

Итоговый Social Score:

\[
S_{\text{social}} =
  \begin{cases}
    50, & \text{если нет ни одного отзыва}, \\
    100 \cdot \frac{\sum\limits_{j} w_j \cdot s^{(\text{rev})}_j}{\sum\limits_{j} w_j}, & \text{иначе}.
  \end{cases}
\]

При добавлении, изменении или удалении отзыва пересчитываются суммы \(\sum w_j\) и \(\sum w_j \cdot s^{(\text{rev})}_j\), после чего обновляется \(S_{\text{social}}\).

\subsection{Итоговый индекс объекта (Health Impact Score 0--100)}

Пусть:

\begin{itemize}
  \item \(S_{\text{infra}} \in [0;100]\) --- инфраструктурный рейтинг;
  \item \(S_{\text{social}} \in [0;100]\) --- рейтинг по отзывам;
  \item \(v_{\text{infra}}\) --- вес инфраструктурной компоненты;
  \item \(v_{\text{social}}\) --- вес компонентной отзывов;
  \item \texttt{verified} --- булев флаг официальной верификации;
  \item \(b_{\text{verified}}\) --- фиксированный бонус за верификацию (например, \(+5\) баллов).
\end{itemize}

\subsubsection{Базовая агрегация}

Пример значений весов (конфигурируемые):

\[
v_{\text{infra}} = 0.7, \quad v_{\text{social}} = 0.3.
\]

Тогда базовый индекс:

\[
S_{\text{base}} = v_{\text{infra}} \cdot S_{\text{infra}} + v_{\text{social}} \cdot S_{\text{social}}.
\]

\subsubsection{Бонус за официальную верификацию}

Если объект имеет статус «верифицирован», применяется бонус:

\[
S_{\text{raw}} =
  \begin{cases}
    S_{\text{base}} + b_{\text{verified}}, & \text{если verified = true}, \\
    S_{\text{base}}, & \text{иначе}.
  \end{cases}
\]

Финальный индекс в диапазоне 0--100:

\[
S_{\text{HIS}} = \min(100, \max(0, S_{\text{raw}})).
\]

\subsection{Обработка отзывов с помощью LLM и связь с анкетой}

\subsubsection{Извлечение фактов из текста}

LLM анализирует текст отзыва и:

\begin{itemize}
  \item извлекает факты об изменении объекта (например, установка новых тренажёров, поломка ограждения, добавление урн, появление рекламы алкоголя);
  \item сопоставляет факты с полями анкеты (по \texttt{field\_id} или синонимам);
  \item формирует предложения по обновлению значений соответствующих полей анкеты.
\end{itemize}

\subsubsection{Механика обновления анкеты через модератора}

\begin{itemize}
  \item При наличии найденных фактов в интерфейсе модератора отображается предложение обновления: ``было = A, предложено = B'' для конкретных полей.
  \item Модератор может:
  \begin{itemize}
    \item \textbf{Подтвердить} --- обновить статическую анкету, пересчитать \(S_{\text{infra}}\) и, следовательно, \(S_{\text{HIS}}\);
    \item \textbf{Запросить уточнение} --- вернуть анкету пользователю с указанием, какие данные необходимо дополнить или подтвердить;
    \item \textbf{Отклонить} --- указать причину (спам, фейк, отсутствие подтверждения и т.п.).
  \end{itemize}
  \item Причины отклонения фиксируются в структурированном виде для дообучения моделей фильтрации и классификации отзывов.
\end{itemize}

\subsubsection{Сентимент-контроль}

LLM сравнивает тональность текста отзыва с выставленной оценкой \(r_j\):

\begin{itemize}
  \item если текст явно позитивный, а оценка низкая (1--2), или наоборот, отзыв помечается как подозрительный;
  \item таким отзывам может автоматически назначаться пониженный вес \(w_{\text{author},j}\) или они направляются в очередь модерации.
\end{itemize}

\subsection{Пересчёт индекса при изменении отзывов}

Модуль должен поддерживать:

\begin{itemize}
  \item добавление отзыва: обновление \(\sum w_j\), \(\sum w_j \cdot s^{(\text{rev})}_j\), пересчёт \(S_{\text{social}}\) и \(S_{\text{HIS}}\);
  \item редактирование отзыва: обновление соответствующего \(r_j\) и времени, пересчёт вкладов;
  \item удаление отзыва: вычитание вклада конкретного отзыва и пересчёт;
  \item периодический учёт time decay (например, ночной batch-job с обновлением весов по времени).
\end{itemize}

При одинаковом наборе анкетных значений и отзывов объект должен получать один и тот же \(S_{\text{HIS}}\) (идемпотентность и воспроизводимость).

\subsection{Обязательные поля данных объекта}

Для каждого объекта в базе данных должны храниться, как минимум:

\begin{itemize}
  \item \texttt{id} --- уникальный идентификатор;
  \item \texttt{type} --- тип объекта (одна из поддерживаемых категорий);
  \item \texttt{address} --- человекочитаемый адрес;
  \item \texttt{latitude}, \texttt{longitude} --- координаты;
  \item \texttt{form\_schema\_id} --- ссылка на JSON-схему анкеты;
  \item \texttt{form\_data} --- фактически заполненные значения полей;
  \item \texttt{S\_infra} --- закэшированный инфраструктурный рейтинг;
  \item \texttt{S\_social} --- закэшированный рейтинг по отзывам;
  \item \texttt{S\_HIS} --- итоговый индекс объекта 0--100;
  \item \texttt{verified} --- флаг официальной верификации, при необходимости дополнительные метаданные о верификаторе.
\end{itemize}
