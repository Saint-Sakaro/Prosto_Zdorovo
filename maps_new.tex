\documentclass[a4paper,12pt]{article}

\usepackage[utf8]{inputenc}
\usepackage[T2A]{fontenc}
\usepackage[russian]{babel}
\usepackage{geometry}
\geometry{margin=2cm}
\usepackage{setspace}
\onehalfspacing

\title{Техническое задание: модуль выбора области и анализа ``здоровости'' в проекте «Карта здоровья»}
\author{}
\date{}

\begin{document}

\maketitle

Настоящее техническое задание описывает модуль выбора области анализа и расчёта индекса «здоровости» территорий в веб-приложении «Карта здоровья». Приложение разрабатывается на стеке Django (backend) и React (frontend) и использует JavaScript API Яндекс.Карт как визуальный слой и инструмент работы с геокоординатами. Все организации (магазины, аптеки, спортивные объекты, заведения общепита и другие точки интереса) полностью хранятся в собственной базе данных приложения. Для каждой организации заранее известен её адрес, который один раз геокодируется через Geocoder API Яндекса в координаты; далее система оперирует только собственными координатами и метаданными объектов.

Цель модуля состоит в том, чтобы предоставить пользователю удобные способы выбора области анализа (радиус вокруг точки, город или округ, улица или квартал), возможность задавать фильтры по типам объектов и получать интегральный рейтинг «здоровости» выбранной зоны, вычисляемый на основе рейтингов всех объектов, попадающих в эту область. Результаты анализа должны визуализироваться на карте, а также использоваться как аналитический инструмент для понимания состояния городской среды. Модуль геймификации (баллы и рейтинг пользователя) на данный модуль напрямую не завязан и работает только на основе пользовательских отзывов.

\section*{Общие принципы: карта и данные}

Отображение карты реализуется с использованием JavaScript API Яндекс.Карт. На карту выводятся только те объекты, которые содержатся в базе данных приложения и имеют сохранённые координаты. Внешние данные о местах, предоставляемые Яндекс.Картами, не сохраняются и не используются как источник справочника. Геокодирование применяется для первичного преобразования адресов в координаты, а также для обратного геокодирования центра выбранной области с целью получения человекочитаемого названия города, района или улицы, отображаемого в интерфейсе.

Фронтенд на React отвечает за инициализацию карты, получение текущего центра, уровня зума и границ видимой области (bounding box), а также за сбор пользовательских настроек (выбор точки, радиуса и фильтров) и формирование запросов к backend. Backend на Django реализует выборку организаций по географическим условиям и фильтрам, рассчитывает интегральные показатели и возвращает результаты анализа в формате JSON.

\section*{Фильтры по типам объектов}

На карте по умолчанию отображаются все организации из базы данных, представленные в виде маркеров. Для работы с большим количеством объектов могут использоваться коллекции и кластеризация. Пользователь может задавать фильтры по типам объектов через набор чекбоксов или переключателей: например, аптеки, места с полезной едой, точки продажи алкоголя и табачной продукции, спортивные объекты, медицинские учреждения и другие категории.

Фильтры действуют на двух уровнях. Во-первых, они управляют визуализацией: на карте отображаются только те маркеры, которые соответствуют активным категориям. Во-вторых, фильтры определяют состав выборки для расчёта индекса «здоровости» области: в расчёт включаются только те объекты, которые удовлетворяют активным фильтрам пользователя. Это позволяет, например, анализировать территорию только по медицинской инфраструктуре, только по точкам здорового питания или по их комбинациям.

\section*{Режим 1: анализ в радиусе окружности}

Режим анализа в радиусе активируется явным действием пользователя. В интерфейсе предусмотрен переключатель «Анализ по радиусу», который включает данный режим независимо от текущего масштаба карты. После активации пользователь выбирает точку-центр анализа кликом по карте или с использованием своей геолокации. Радиус анализа задаётся через слайдер или текстовое поле (в метрах или километрах).

Фронтенд формирует запрос к backend, содержащий координаты центра (широта и долгота), значение радиуса и текущую конфигурацию фильтров по типам объектов. Backend на Django выбирает из базы данных все организации, которые одновременно находятся в пределах заданного радиуса от центра и соответствуют активным фильтрам. Для выбранного множества объектов рассчитывается интегральный рейтинг зоны и сводная статистика: количество объектов по категориям, их плотность, расстояния до ключевых типов инфраструктуры и другие показатели.

Для каждой организации в базе данных хранится собственный рейтинг «здоровости» (или набор параметров, сводимых к единому числу), который формируется на основе пользовательских оценок, отзывов и внутренних характеристик объекта. Интегральный рейтинг зоны определяется как агрегированная функция от рейтингов всех учтённых объектов, например средневзвешенное значение, где весами могут выступать тип объекта, его важность для здоровья, объём пользовательской активности или расстояние до центра области. Таким образом, вклад каждого объекта в индекс зоны пропорционален его индивидуальному рейтингу и значимости.

Результаты анализа включают числовой индекс «здоровости» зоны в диапазоне от 0 до 100, текстовую интерпретацию (например, неблагоприятная, средняя или благополучная зона) и агрегированные показатели по категориям объектов. На карте анализируемая область выделяется в виде окружности, внутри которой отображаются только те маркеры, которые вошли в расчёт индекса. Фон зоны может подсвечиваться полупрозрачным зелёным оттенком, интенсивность которого растёт вместе с величиной индекса, в то время как остальная часть карты остаётся в базовом стиле.

\section*{Режим 2: анализ по городу или округу}

Во втором режиме анализ выполняется на уровне города или крупного округа. Этот режим применяется, когда переключатель «анализ по радиусу» выключен, а масштаб карты соответствует обзору города или административного района. В таком случае видимая область карты (bounding box) трактуется как выбранная пользователем территория.

Фронтенд периодически либо по нажатию на кнопку «Посчитать индекс для текущей области» считывает координаты юго-западного и северо-восточного углов видимой части карты. Эти координаты передаются на backend вместе с информацией о применённых фильтрах. Дополнительно по центру карты может выполняться обратное геокодирование для получения названия города или округа; это название используется только для отображения в интерфейсе и не влияет на выборку.

Backend, получив bounding box и фильтры, выбирает из базы данных все организации, чьи координаты попадают внутрь указанного прямоугольника и соответствуют активным фильтрам. Далее на основе рейтингов этих объектов вычисляется интегральный индекс «здоровости» для всей видимой области, а также формируется статистика по категориям: количество аптек, мест здорового питания, точек продажи алкоголя и табака, спортивных объектов и т.п. Итоговый индекс интерпретируется как оценка «здоровости» города или округа.

На карте для визуализации результата может использоваться полупрозрачный прямоугольный полигон по границам видимой области, который подсвечивается зелёным цветом с интенсивностью, зависящей от величины индекса. Внутри этого полигона остаются видимыми только те маркеры, которые входят в выборку анализа (по области и фильтрам), остальная часть карты по-прежнему показывает все доступные объекты с учётом глобальных фильтров.

\section*{Режим 3: анализ по улице или кварталу}

Третий режим применяется для детального анализа на уровне улицы или небольшого квартала. Он активируется автоматически, когда переключатель «анализ по радиусу» выключен, а масштаб карты увеличен до уровня, на котором пользователь рассматривает относительно узкую территорию (квартал, группа домов, отдельная улица). В этом случае видимая часть карты, описываемая bounding box, интерпретируется как локальная область анализа.

Фронтенд получает координаты границ видимой области и координаты центра карты. По центру может выполняться обратное геокодирование для определения названия улицы, которое затем используется в пользовательском интерфейсе (например, «Анализ по улице: Тверская»). Выборка организаций на backend снова осуществляется строго по координатам bounding box и активным фильтрам, а не по текстовому названию улицы.

Backend выбирает из базы все организации, расположенные внутри этого небольшого прямоугольника и подходящие под фильтры, затем рассчитывает интегральный индекс «здоровости» по тем же принципам, что и в других режимах: как агрегированную функцию от рейтингов всех входящих в область объектов. Дополнительно могут рассчитываться показатели плотности определённых типов объектов, среднее расстояние до ближайших аптек или спортивных площадок и другие локальные метрики.

На карте локальная зона выделяется узким прямоугольником или иным контуром, а её фон подсвечивается в зависимости от величины индекса. Внутри выделенной области отображаются только те маркеры, которые были учтены в расчёте, что позволяет пользователю визуально связать числовое значение индекса с конкретными объектами инфраструктуры.

\section*{Правила выбора режима по масштабу}

Выбор режима анализа сочетает явное управление и автоматическое определение по масштабу карты. Если пользователь активировал режим «анализ по радиусу», всегда используется режим радиуса, и запросы на backend формируются только в формате «центр + радиус + фильтры». В этом случае масштаб карты влияет лишь на визуальное представление, но не на тип анализа.

Если режим радиуса выключен, тип анализа определяется текущим уровнем зума карты. В конфигурации фронтенда задаются пороговые значения зума: диапазон для уровня города/округа и диапазон для уровня улицы/квартала. Если текущий зум попадает в городской диапазон, применяется режим анализа по городу или округу; если зум превышает установленный порог и карта приближена до уровня улицы, используется режим анализа по улице или кварталу. Эти пороги должны быть настраиваемыми, чтобы их можно было адаптировать под особенности различных городов и плотность застройки без изменения серверной логики.

\section*{Интеракция с объектами на карте}

Для каждого объекта инфраструктуры, отображаемого на карте (аптека, магазин, спортивный объект и т.п.), должна быть реализована интерактивная карточка с информацией и возможностью оставить отзыв. При клике пользователя по маркеру объекта на карте открывается всплывающее окно (balloon или панель) с краткими данными об объекте: названием, типом, адресом, базовыми «здоровостными» характеристиками и текущим рейтингом объекта, сформированным на основе пользовательских оценок и отзывов.

Во всплывающем окне должна присутствовать явная кнопка или ссылка «Оставить отзыв», по нажатию на которую пользователю отображается форма создания отзыва, уже привязанная к выбранному объекту. Форма отзыва может открываться непосредственно внутри балуна, в боковой панели или в модальном окне, но во всех случаях идентификатор объекта и его основные данные должны подставляться автоматически, без необходимости повторного выбора точки. После успешной отправки отзыва интерфейс обновляет отображаемый рейтинг объекта и, при необходимости, связанные с объектом агрегированные показатели, используемые при расчёте индекса «здоровости» для соответствующих областей.

\section*{Требования к фронтенду}

Фронтенд (React + JavaScript API Яндекс.Карт) должен:

\begin{itemize}
  \item Инициализировать карту, получать центр, уровень зума и границы видимой области (bounds).
  \item Отрисовывать все доступные объекты из базы данных в виде маркеров, с учётом активных фильтров по типам.
  \item Предоставлять интерфейс для управления фильтрами: чекбоксы или переключатели по категориям объектов.
  \item Реализовать переключатель «анализ по радиусу», слайдер/поле ввода радиуса и кнопку запуска анализа для текущей области.
  \item В зависимости от режима формировать запрос к backend:
  \begin{itemize}
    \item режим радиуса: центр, радиус, набор активных фильтров;
    \item режим города/округа: bounding box видимой области, фильтры;
    \item режим улицы/квартала: bounding box, фильтры, при необходимости название улицы, полученное через обратное геокодирование центра.
  \end{itemize}
  \item Визуализировать результаты анализа: подсветку области (круг или полигон), числовое значение индекса «здоровости», текстовую интерпретацию и сводную статистику по категориям.
  \item Реализовать интерактивные карточки объектов: при клике по маркеру показывать информацию об объекте и кнопку/форму для создания отзыва.
\end{itemize}

\section*{Требования к backend}

Backend (Django) должен реализовать единый HTTP-эндпоинт анализа области, который:

\begin{itemize}
  \item Принимает запросы с параметрами:
  \begin{itemize}
    \item для режима радиуса: широта и долгота центра, радиус, список активных фильтров;
    \item для режимов города/округа и улицы/квартала: координаты bounding box (юго-западный и северо-восточный углы), список активных фильтров, опционально тип области.
  \end{itemize}
  \item Определяет режим анализа и выполняет выборку организаций из базы:
  \begin{itemize}
    \item для радиуса: по расстоянию до центра;
    \item для города/округа и улицы/квартала: по попаданию координат в bounding box.
  \end{itemize}
  \item Фильтрует выборку по типам объектов в соответствии с активными фильтрами пользователя.
  \item Рассчитывает интегральный индекс «здоровости» области на основе рейтингов всех учтённых объектов, используя согласованный метод агрегации (например, средневзвешенное с учётом типа и значимости объекта).
  \item Формирует и возвращает JSON с полями:
  \begin{itemize}
    \item числовой индекс здоровья зоны;
    \item тип анализируемой области (радиус, город, улица);
    \item человекочитаемая подпись (название города или улицы, если определено);
    \item статистика по категориям объектов;
    \item список точек с базовыми параметрами, использованных в расчёте.
  \end{itemize}
\end{itemize}

\section*{Интеграция с геймификацией}

Модуль анализа областей «здоровости» логически связан с общим контекстом приложения, но не влияет напрямую на начисление баллов и рейтинга пользователя. Система геймификации основана исключительно на пользовательских отзывах и связанных с ними действиях: создании отзывов о точках инфраструктуры, фиксации инцидентов, прикреплении доказательных материалов (фото, видео), прохождении модерации и подтверждении уникальных и полезных сообщений.

Баллы (виртуальная валюта) и рейтинг пользователя (репутация, месячный рейтинг) начисляются только за работу с отзывами в соответствии с правилами модуля геймификации: минимальные награды за дублирующие подтверждения, отложенные и повышенные награды за потенциально уникальные инциденты после модерации, штрафы за спам и фейковые сообщения. Результаты анализа областей на карте используются только как аналитический и визуальный инструмент для понимания «здоровости» территории и не дают пользователю дополнительных очков или изменений рейтинга.

\end{document}
